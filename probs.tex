\documentclass[12pt]{article}
\usepackage{a4wide}
\usepackage{amsmath}

\begin{document}

\noindent
\textbf{Problem 1}

\noindent
(1) Translate the given image by \((t_x=3.75,t_y=4.3)\) pixels. Use 
target-to-source mapping.
\begin{enumerate}
\item for every target pixel \((x_t,y_t)\)
\item find the source pixel \((x_s,y_s)\) = \((x_t,y_t) - (t_x,t_y)\)
\item use round-off or bilinear interpolation to locate an integer
value for the calculated source pixel
\end{enumerate}

\noindent
(2) Rotate the given image by 2.5 and -2.5 degrees. Use 
target-to-source mapping.
\begin{enumerate}
\item for every target pixel \((x_t,y_t)\)
\item find the source pixel \((x_s,y_s) = 
(x_t \cos{\theta} - y_t \sin{\theta}, 
x_t \sin{\theta} + y_t \cos{\theta},)\)
\item use round-off or bilinear interpolation to locate an integer
value for the calculated source pixel
\end{enumerate}

\noindent
\textbf{Problem 2}

\noindent
(1) Scale the given image by 0.8 and 1.3 factors. Use 
target-to-source mapping.
\begin{enumerate}
\item for every target pixel \((x_t,y_t)\)
\item find the source pixel \((x_s,y_s)\) = \((x_t/a,y_t/a)\)
\item use round-off or bilinear interpolation to locate an integer
value for the calculated source pixel
\end{enumerate}

\noindent
(2) Convolve the given image with Gaussian kernel of these 
\(\sigma\) values : 0.8, 1.2 and 1.6.
\begin{enumerate}
\item find the support (kernel size) for the given \(\sigma\).
\[\text{support} = \text{ceil}(6 \sigma + 1)\] support must be odd.
\item create a Gaussian kernel
\[\frac{1}{2 \pi \sigma^2} 
\exp{\bigg(-\frac{m^2+n^2}{2 \sigma^2}\bigg)}\]
\item convolve: move kernel over each pixel, perform weighted average 
and assign it to target image
\end{enumerate}

\noindent
\textbf{Problem 3}

\noindent
Apply a space variant blur to the given image.
\begin{enumerate}
\item create a blur surface
\[\sigma(m,n) = 
\frac{1}{2 \pi \sigma_t^2} 
\exp{\bigg(-\frac{(m-\frac{N}{2})^2+(n-\frac{N}{2})^2}
{2 \sigma_t^2}\bigg)}\]
\item find the support (kernel size) for each pixel 
(corresponding \(\sigma\))
\[\text{support} = \text{ceil}(6 \sigma + 1)\] support must be odd.
\item create a Gaussian kernel
\[\frac{1}{2 \pi \sigma^2} 
\exp{\bigg(-\frac{m^2+n^2}{2 \sigma^2}\bigg)}\]
\item multiply kernel with the pixel intensity value, keep it as a 
layer image
\item repeat for all pixels, add all layers
\end{enumerate}

\noindent
\textbf{Problem 4}

Deduce the shape of the object from the given set of frames
\begin{enumerate}
\item for every pixel
\item form an array of values calculated from focus operator
\[SML(g(x,y)) = \sum_{i=x-q}^{x+q} \sum_{j=x-q}^{x+q} ML(g(i,j))\]
\[ML(g(x,y)) = |g_{xx}| + |g_{yy}| \]
\item calculate the distance of that pixel (\(\overline{d}\))
from the focus plane using
 the  peak value, its previous and next values (see notes for formula)
\item form the shape by using the \(\overline{d}\) of all pixels
\end{enumerate}

\noindent
\textbf{Problem 5}

\noindent
(1) Compute 2D DFT from 1D DFT
\begin{enumerate}
\item compute 1D DFT of each row of the image
\item compute 1D DFT of each column of the DFT image
\item transpose
\end{enumerate}

\noindent
(2) Interchange phase components of DFT of two images, and 
observe their IDFTs
\begin{enumerate}
\item Phase component contains the high frequency information 
(edges) and hence is responsible for the visual appearance.
\end{enumerate}

\noindent
\textbf{Problem 6}

\noindent
Singular value decomposition
\begin{enumerate}
\item Compute SVD.  
\item Drop eigen values one-by-one from the decomposition 
and observe the output image (compression).  
\item Calculate the error.
\end{enumerate}


\noindent
\textbf{Problem 7}
Bilateral filter the image
\end{document}
